\documentclass{article}
\usepackage{geometry}
\usepackage{hyperref}
\usepackage{array}
\usepackage{ragged2e}
\usepackage{enumitem}
\usepackage{microtype}
\usepackage{parskip}
\pagestyle{empty}
\setlength{\parskip}{1ex}
\hypersetup{colorlinks=true}
\geometry{a4paper, margin=1in}

\begin{document}


\begin{center}
    \textbf{\Large Hankun Xu}
\end{center}

\begin{center}
    Email: uint16.kun@gmail.com \textbar \  Homepage: \href{https://resume.uint16kun.com/}{https://resume.uint16kun.com/}
\end{center}

\section*{Education}
\makebox[\textwidth][l]{
    \textbf{Sixian Mlddie School}, Zhengzhou, China \hfill \textbf{2015.9 - 2018.6}%
}
\makebox[\textwidth][l]{
    \textbf{Zhengzhou Experimental High School}, Zhengzhou, China \hfill \textbf{2018.9 - 2021.6}%
}
\makebox[\textwidth][l]{
    \textbf{Huzhou University}, Huzhou, China \hfill \textbf{2021.9 - 2025.6}%
}
\begin{itemize}
    \item Major: Electronic Information Engineering
    \item GPA: 4.00/5
    \item Core courses: Digital Logic Circuits (92), Analog Electronic Technology (90), Electromagnetic Fields and Waves (85), Principles of Automatic Control (92), Communication Circuits (90), Advanced Language Programming (92), Circuit Analysis (86), Digital Signal Processing (86)
\end{itemize}

\section*{Awards}
\subsection*{Scholarship}
\begin{itemize}
    \item 2021-2022 First class scholarship of the school
    \item 2022-2023 School Special Scholarship
\end{itemize}

\subsection*{Subject competition}
\begin{itemize}
    \item 2022 \textbf{TI Cup} National Undergraduate Electronic Design Contest \textbf{Provincial Third Prize (TOP 40\%)}
    \item 2023 \textbf{TI Cup} National Undergraduate Electronic Design Contest \textbf{National Second Prize (TOP 8\%)}
    \item 2024 \textbf{Renesas Cup} National College Student Electronic Design Contest Information Technology Frontier Special Contest \textbf{First Prize in Eastern Division}
    \item 2024 \textbf{Renesas Cup} National College Student Electronic Design Contest Information Technology Frontier Special Contest \textbf{National Third Prize}
\end{itemize}

\section*{Research}
\begin{itemize}
    \item Machine vision
    \item Automatic control system
    \item Embedded system design
    \item Digital/analog circuit design
    \item Front-end/back-end development
    \item Internet of Things system development
\end{itemize}

\section*{Project}
\begin{itemize}
    \item \href{https://resume.uint16kun.com/my-projects/Moving%20target%20control%20and%20automatic%20tracking%20system.html}{Moving target control and automatic tracking system} (2023.8)     Advisor:\href{https://xxgcxy.zjhu.edu.cn/2023/0411/c5546a193663/page.htm}{Lili Yao}
          \begin{itemize}
              \item[-] A two-dimensional platform built using NVIDIA Jetson platform, OpenCV computer vision library, and brushless motors to achieve real-time detection, tracking, and control of moving targets.
          \end{itemize}
    \item \href{https://resume.uint16kun.com/my-projects/High-throughput%20phenotyping%20system%20for%20potted%20plants.html}{High-throughput phenotyping system for potted plants} (2024.4)     Advisor:\href{https://xxgcxy.zjhu.edu.cn/2021/0326/c5544a166633/page.htm}{Xiangxiang Fan}    
          \begin{itemize}
              \item[-] A two-dimensional rotating platform using stepper motors and a single-point laser ranging module for 3D modeling of plants, analyzing plant growth through spectral analysis, and displaying data on a web page developed using the Vue framework.
          \end{itemize}
    % \item \href{https://resume.uint16kun.com/my-projects/Smart%20medicine%20delivery%20car.html}{Smart medicine delivery car}(2022.5)     Advisor:\href{https://xxgcxy.zjhu.edu.cn/2021/0326/c5544a166633/page.htm}{Xiangxiang Fan}
    %       \begin{itemize}
    %           \item[-] A microcontroller running a model trained with the YOLOv3 algorithm to recognize room numbers, simulating the delivery and pickup of medicines between hospital pharmacies and patient rooms.
    %       \end{itemize}
    % \item \href{https://resume.uint16kun.com/my-projects/Rolling%20ball%20control%20system.html}{Rolling ball control system}(2022.7)     Advisor:\href{https://xxgcxy.zjhu.edu.cn/2021/0326/c5544a166633/page.htm}{Xiangxiang Fan}
    %       \begin{itemize}
    %           \item[-] Using the OpenMV library in STM32H7 to identify the ball, monitor the position of the rolling ball in real-time, and control the tilt angle of the plate with servos using a PID algorithm to control the ball's position.
    %       \end{itemize}
    \item \href{https://resume.uint16kun.com/my-projects/Car%20following%20driving%20system.html}{Car following driving system} (2022.7)     Advisor:\href{https://xxgcxy.zjhu.edu.cn/2021/0326/c5544a166633/page.htm}{Xiangxiang Fan}
          \begin{itemize}
              \item[-] Line tracking through infrared sensors, distance monitoring through UWB modules, data exchange through wireless serial communication modules, and distance control through a PID algorithm.
          \end{itemize}
    \item \href{https://resume.uint16kun.com/my-projects/Non-contact%20object%20size%20and%20shape%20measurement.html}{Non-contact object size and shape measurement} (2023.5)     Advisor:\href{https://xxgcxy.zjhu.edu.cn/2021/0326/c5544a166633/page.htm}{Xiangxiang Fan}
          \begin{itemize}
              \item[-] Analyzing the shape and pixel length of graphic edges through a camera, calculating geometric parameters after obtaining the distance through a 2D pan-tilt and laser ranging.
          \end{itemize}
    \item \href{https://resume.uint16kun.com/my-projects/Brushless%20motor%20drive%20circuit%20and%20FOC%20control%20algorithm%20design.html}{Brushless motor drive circuit and FOC control algorithm design} (2023.4)
          \begin{itemize}
              \item[-] Designing a three-phase full-bridge drive circuit for brushless motors and controlling it using the FOC algorithm.
          \end{itemize}
\end{itemize}

\section*{Skill}
\subsection*{Programming Language:}
C, Python, C++, JavaScript, Verilog, MATLAB, Lua

\subsection*{Software:}
Keil, STM32CubeMX, LCEDA, Altium Designer, SolidWorks, LabVIEW, Multisim, Visual Studio Code, Git, Markdown, LaTeX, Anaconda, 
Xshell, Xftp, Quartus II
\subsection*{Language:}
TOEIC 690

\end{document}